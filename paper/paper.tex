\documentclass{article}

\usepackage[style=numeric]{biblatex} % Bibliography style
\usepackage[colorlinks=true]{hyperref} % Href
\usepackage[parfill]{parskip} % New lines between paragraphs
\usepackage{csquotes} % Block quoting

\title{CS7641 Assignment 1 --- Supervised Learning}
\author{Phil Condreay (\href{mailto:astex@gatech.edu}{astex@gatech.edu})}

\bibliography{paper}

\begin{document}

  \maketitle

  \begin{abstract}

    Machine learning is a technology that is often publicly misunderstood,
    both narrower in the problem spaces that it can address and broader in the
    variety and completeness with which it addresses them. Technologists often
    espouse that machine learning can acheive anything and solve any problem.
    This is not the case. And, agents often require a careful guiding human
    hand to come to useful conclusions. This belies the complete ability of
    machine learning to solve certain subclasses of problems to a degree that
    the same technologists can rarely imagine. In this paper, I explore this
    duality by applying supervised learning algorithms to two interesting
    problem spaces.

  \end{abstract}

  \section*{Introduction}\label{s:intro}

  In order to explore these problems, I need a working implementation of
  various machine learning algorithms. My background in python and some amount
  of internet searching lead me to scikit-learn~\cite{scikit-learn}. This
  library implements decision trees, neural networks, gradient-descent
  boosting, support vector machines, k-nearest neighbors, and a number of other
  algorithms that will not be used in this paper. Plotly~\cite{plotly} will be
  used to generate graphs because it has the shortest URL.\

  \printbibliography{}

\end{document}
